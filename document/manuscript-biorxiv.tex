\documentclass[10pt,letterpaper]{article}
\usepackage[top=0.85in,left=2.75in,footskip=0.75in,marginparwidth=2in]{geometry}

% use Unicode characters - try changing the option if you run into troubles with special characters (e.g. umlauts)
\usepackage[utf8]{inputenc}

% clean citations
\usepackage{cite}

% hyperref makes references clicky. use \url{www.example.com} or \href{www.example.com}{description} to add a clicky url
\usepackage{nameref,hyperref}

% line numbers
\usepackage[right]{lineno}

% improves typesetting in LaTeX
\usepackage{microtype}
\DisableLigatures[f]{encoding = *, family = * }

% text layout - change as needed
\raggedright
\setlength{\parindent}{0.5cm}
\textwidth 5.25in 
\textheight 8.75in

% Remove % for double line spacing
%\usepackage{setspace} 
%\doublespacing

% use adjustwidth environment to exceed text width (see examples in text)
\usepackage{changepage}

% adjust caption style
\usepackage[aboveskip=1pt,labelfont=bf,labelsep=period,singlelinecheck=off]{caption}

% remove brackets from references
\makeatletter
\renewcommand{\@biblabel}[1]{\quad#1.}
\makeatother

% headrule, footrule and page numbers
\usepackage{lastpage,fancyhdr,graphicx}
\usepackage{epstopdf}
\pagestyle{myheadings}
\pagestyle{fancy}
\fancyhf{}
\rfoot{\thepage/\pageref{LastPage}}
\renewcommand{\footrule}{\hrule height 2pt \vspace{2mm}}
\fancyheadoffset[L]{2.25in}
\fancyfootoffset[L]{2.25in}

% use \textcolor{color}{text} for colored text (e.g. highlight to-do areas)
\usepackage{color}

% define custom colors (this one is for figure captions)
\definecolor{Gray}{gray}{.25}

% this is required to include graphics
\usepackage{graphicx}

% use if you want to put caption to the side of the figure - see example in text
\usepackage{sidecap}

% use for have text wrap around figures
\usepackage{wrapfig}
\usepackage[pscoord]{eso-pic}
\usepackage[fulladjust]{marginnote}
\reversemarginpar

% document begins here
\begin{document}
\vspace*{0.35in}

% title goes here:
\begin{flushleft}
{\Large
\textbf\newline{Open Science in Kenya: Where are we?}
}
\newline
% authors go here:
\\
Author 1\textsuperscript{1},
Author 2\textsuperscript{2},
Author 3\textsuperscript{1},
Author 4\textsuperscript{1},
Author 5\textsuperscript{2},
Author 6\textsuperscript{2},
Author 7\textsuperscript{1,*}
\\
\bigskip
\bf{1} Affiliation A
\\
\bf{2} Affiliation B
\\
\bigskip
* correseponding@author.mail

\end{flushleft}

\section*{Abstract}
Fitting multiple figures into very tight manuscripts while keeping it pleasant to read is challenging. Therefore figures are often simply attached to the very end of a manuscript file. While easier for the authors, this practice is inconvenient for readers. This \LaTeX template shows how to generate a compiled PDF with figures embedded into the text. It provides several examples of how to embed figures or tables directly into the text thus giving you a range of options from which you should choose the one best suited for your manuscript. Check out Schlegel et al., (2016) as example of use \cite{Schlegel2016}.

% now start line numbers
\linenumbers

\input manuscript

% the * after section prevents numbering

\section*{Acknowledgments}
We thank KENET for providing us with an ample environment for our hackathon. 

\nolinenumbers

%This is where your bibliography is generated. Make sure that your .bib file is actually called library.bib
\bibliography{library}

%This defines the bibliographies style. Search online for a list of available styles.
\bibliographystyle{abbrv}

\end{document}

